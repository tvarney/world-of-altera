\documentclass[a4paper, 11pt]{article}
\usepackage[margin=.8in]{geometry}
\usepackage{times}
\usepackage{hyperref}

% It might make things easier to define \level like so:
% \def\level#2{\item[(#1)\hspace{9.5pt}]\begin{itemize}{#2}\end{itemize}}
% or something to that effect.
\def\level#1{\item[(#1)\hspace{9.5pt}]}
\newenvironment{knowlevels}{\begin{description}}{\end{description}} % Just an alias.
\def\decadeblk#1{\subsection{#1}}
\def\yearblk#1{\subsubsection{#1}}

\begin{document}
\title{History of Modern Altera \\ \large{Timeline of Events}}
\date{\today}
\author{Isami Romanowski \and Troy Varney}
\maketitle

\tableofcontents

\section{Preface}

\section{Timeline}
Events are listed per-year in grouped bulleted form.  Bullet groups correspond to approximately what level of education or knowledge is required for a person in present-day Altera to be privy to the bulleted knowledge:
\begin{knowlevels}
\level{1}  Almost anyone, regardless of education level, would likely know this information.

\level{2}  Anyone with an approximately university-level education, or who is well-read in history, would likely know this information.

\level{3}  Anyone with a history specialization at the university level (i.e. history majors, PhDs, etc.) or equivalent would likely know this information.

\level{4}  Only educated historians with a specialization in the particular era in question, or people who were somehow related to the event (either through direct participation, or through a small degree of separation from those who participated) would likely know this information.
\end{knowlevels}
Years which lacked major world events are ommitted from the listing.  Years are grouped by decade.

\decadeblk{430s B.C}
\yearblk{439 B.C.}
\begin{knowlevels}
\level{2} {
  \begin{itemize}
  \item The Etraean Republic is founded.
  \end{itemize}
}
\end{knowlevels}

\decadeblk{0s A.C.}
\yearblk{1 A.C.}
\begin{knowlevels}
\level{2} {
\begin{itemize}
\item The Etraean general Gaius Carantus returns to the Etraean capital with his army after his successful military campaign in the Eastern Galanos.  Upon his return, he and his army seize the Senate while in session, and Carantus demands to be made the sole ruler.  The Senate, led in part by Consul Titus Carantus, Gaius' father, refuses to bow to his demands.  Gaius Carantus orders the massacre of all present members of the Senate, thereby overthrowing the Etraean Republic.  Carantus declares himself ruler of Etraea, and bestows upon himself the title of Rex.

\item The Etraean capital is renamed to Carantium, and the Etraean Republic becomes known as the Carantine Empire.

\item The Age of Carantus (A.C.) begins.

\item An advisory council to the Carantine king is formed.  Initial members of the council are the favored commanders who served directly under Rex Gaius Carantus.  Military participation is a prerequisite to holding a council position.
\end{itemize}
}
\end{knowlevels}

\yearblk{2 A.C.}
\begin{knowlevels}
\level{2} {
\begin{itemize}
\item  The Imperial Registry is established in the Carantine Empire.  The Registry takes over administration of the national census, which was regularly taken during the days of the Etraean Republic.  A new census is taken, and each recorded person in the Empire is issued paper identification documents by the Registry.  Such identification was not present during the days of the Republic.  Identification documents include details such as race, social class at the time of issuance, and magical capability.  The process of taking census and issuing the new identification papers takes some two years.  Mages assist greatly with the increased bookkeeping burden caused by the new identification records.
\end{itemize}
}
\end{knowlevels}

\decadeblk{10s A.C.}
\decadeblk{20s A.C.}
\decadeblk{30s A.C.}

\decadeblk{1200s A.C./0s D.R.}
\yearblk{1207 A.C/0 D.R}
\begin{knowlevels}
\level{1} {
\begin{itemize}
\item The Mage Society tricks the current Emperor into signing over control of the empire to the advisory council in return for reclaiming the lost Carantine lands.
\item The Two Weeks War begins and lasts 13 days, during which time the Mage Society conquers most of the known world. 
\end{itemize}
}
\end{knowlevels}

\decadeblk{140s D.R.}
\yearblk{141 D.R}
\begin{knowlevels}
\level{3} {
\begin{itemize}
\item The Diviners and Illusionists seceed from the Mage Dominion due to discontent with their increasing marginalization by the Evokers and Shapers.
\item The Evokers condemn this action, and declare that any who stand against them shall be destroyed.
\end{itemize}
}
\level{2} {
\begin{itemize}
\item The Mage Wars begin.
\end{itemize}
}
\end{knowlevels}

\yearblk{142 D.R}
\begin{knowlevels}
\level{3} {
\begin{itemize}
\item The Enchanters side with the Illusionists and Diviners. 
\end{itemize}
}
\end{knowlevels}

\yearblk{143 D.R}
\begin{knowlevels}
\level{3} {
\begin{itemize}
\item The Conjurers officially denounce the actions of Illusion and Divination schools.
\end{itemize}
}
\end{knowlevels}

\yearblk{144 D.R}
\begin{knowlevels}
\level{3} {
\begin{itemize}
\item The theory behind a spell that creates a feedback-loop is perfected by the Evokers. Work begins on creating the sphere of destruction.
\end{itemize}
}
\end{knowlevels}

\yearblk{147 D.R}
\begin{knowlevels}
\level{3} {
\begin{itemize}
\item The Evokers finish the Sphere of Destruction spell.
\end{itemize}
}
\level{1} {
\begin{itemize}
\item The Evokers use the Sphere of Destruction against a major city held by their opposition. An area with a radius of 100 miles is drained of all magic. Everything in the area of the spell is killed.
\end{itemize}
}
\level{3} {
\begin{itemize}
\item The Conjurers and Shapers debate the Evokers about the morality of such a spell.
\end{itemize}
}
\end{knowlevels}

\yearblk{148 D.R}
\begin{knowlevels}
\level{2} {
\begin{itemize}
\item The Evokers begin using the Sphere of Destruction with regularity. To speed up the rate at which they can use it, destroying cities and garrisons.
\end{itemize}
}
\end{knowlevels}

\yearblk{149 D.R}
\begin{knowlevels}
\level{3} {
\begin{itemize}
\item The Shapers and Conjurers denounce the Evokers actions.
\end{itemize}
}
\level{2} {
\begin{itemize}
\item The Shapers move the bulk of their troops to the southern continent acros the isthmus.
\end{itemize}
}
\end{knowlevels}

\decadeblk{150s D.R}
\yearblk{150 D.R}

\end{document}
